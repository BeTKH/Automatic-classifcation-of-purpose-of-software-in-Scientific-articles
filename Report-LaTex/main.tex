% Instead of "de", "en" can also be used here if the work is written in English
% Either "darkstyle" or "lightstyle"
\documentclass[en, lightstyle]{unirostock}

% Für die Quellenangaben, weitere Informationen https://de.overleaf.com/learn/latex/Bibliography_management_with_biblatex
% For the references, further information https://de.overleaf.com/learn/latex/Bibliography_management_with_biblatex

\usepackage[backend=biber,style=alphabetic,maxbibnames=99,backref=true,citestyle=alphabetic]{biblatex}
\addbibresource{lit.bib}

% Welche Bedingungen gibt es für die Verbreitung der Arbeit?
% creative-commons: Creative Commons Lizenz, Namensnennung - Weitergabe erlaubt unter gleichen Bedingungen 
% private: Veröffentlichung und Veränderung nur nach Rücksprache mit dem Autor

%ENGLISH
% What are the conditions for the dissemination of the work?
% creative-commons: Creative Commons license, attribution - redistribution permitted under alike
% private: Publication and changes only after consultation with the author

\license{creative-commons}

\author{Kobro Bekalue}
\enrolmentnumber{218205220} % Matrikelnummer  (Matriculation number)

\title{Automatic classification of software usage statements}
\type{Masters Thesis}

\course{Electrical Engineering - ( Specialization IT ) }
\workperiod{08. November 2021 -- 21. March 2022}
\supervisor{Dr. Frank Krüger and David Schindler}
\primaryreviewer{Prof. Sacha Spors}
%\secondaryreviewer{Prof. Dr. rer. nat. habil. Andreas Heuer} % Falls es keinen gibt, einfach weglassen  (If there isn't one, just leave it out )

\faculty{Faculty of Computer Science and Electrical Engineering}
\institute{Institute of Computer Science}
\workinggroup{}

\begin{document}
\maketitle
\makelicense

\pagenumbering{Roman}



\setstretch{1.263}
\pagenumbering{gobble}

% Declaration
% =========================================
{\normalfont
\color{uniblau}
\huge\sffamily\itshape
Declaration
}

Lorem ipsum dolor sit amet, consectetur adipiscing elit. Nunc non pulvinar enim. Maecenas euismod odio non velit iaculis laoreet. Ut iaculis sapien eget libero pretium consectetur. Pellentesque ullamcorper accumsan libero sed sagittis. Pellentesque ullamcorper ante id nibh ullamcorper porttitor. Ut in urna velit. Donec commodo, tortor quis tempor fringilla, est justo aliquam enim, vel malesuada eros mauris eget velit. Duis convallis quis arcu at mollis. 

\today \\Rostock, Germany
\\
 \\
\\
\begin{Form}
  \digsigfield{7cm}{3cm}{Unterschrift}
\end{Form}

\vfill
\clearpage

% Aknowledgements
% =========================================
%  \vspace{15cm}   make sapce on the top

{\normalfont
\color{uniblau}
\huge\sffamily\itshape
Acknowledgement
}

I would like Lorem ipsum dolor sit amet, consectetur adipiscing elit. Nunc non pulvinar enim. Maecenas euismod odio non velit iaculis laoreet. Ut iaculis sapien eget libero pretium consectetur. Pellentesque ullamcorper accumsan libero sed sagittis. Pellentesque ullamcorper ante id nibh ullamcorper porttitor. Ut in urna velit. Donec commodo, tortor quis tempor fringilla, est justo aliquam enim, vel malesuada eros mauris eget velit. Duis convallis quis arcu at mollis. Nulla tempor sed nisi sed pellentesque. Nam felis libero, auctor ac dignissim quis, pellentesque sed tortor. Aliquam lobortis ex maximus varius bibendum. Duis rutrum, neque sit amet blandit rhoncus, orci lectus sagittis purus, et finibus libero dui molestie quam. Phasellus pulvinar tempus justo quis iaculis. Sed vehicula elit sodales sagittis posuere. Aliquam quis diam leo. Pellentesque rutrum orci at ex sagittis, ut pharetra lectus hendrerit.

Vestibulum ante ipsum primis in faucibus orci luctus et ultrices posuere cubilia curae; Nunc nec pulvinar nisl. Integer mollis at tortor eu tincidunt. Maecenas elit arcu, iaculis eget libero vitae, feugiat commodo diam. Donec at dapibus libero. Curabitur in dui luctus, aliquam neque ut, rhoncus enim. Aenean est mi, vehicula non lectus ac, euismod luctus enim. Proin eu euismod enim, id feugiat justo. Phasellus iaculis gravida sollicitudin. Nullam sit amet massa metus. Vestibulum sagittis nulla ante, nec consequat libero congue eu. Vestibulum vel lectus quis sapien laoreet vulputate a eget odio. Suspendisse potenti. Nunc pellentesque ante nec nunc condimentum euismod. Nullam sodales leo ex, non sagittis eros laoreet id. Cras vehicula ultricies lobortis.

Nullam sodales blandit odio, at tempus est gravida vitae. Morbi id justo ante. Mauris ut nulla blandit, iaculis lectus at, dictum lacus. Fusce scelerisque justo ut ipsum semper fringilla vitae ac nisi. Pellentesque eget elementum nisi. Duis at blandit dui. Nulla sed odio eget turpis tristique ullamcorper quis vel diam. Nullam odio felis, volutpat in ex vel, pellentesque fringilla lacus. Nulla facilisi. Aliquam auctor consectetur magna, eget venenatis urna blandit eget. Maecenas tincidunt risus lorem, eu suscipit lacus fringilla sed. Pellentesque habitant morbi tristique senectus et netus et malesuada fames ac turpis egestas. Integer a pellentesque felis, non cursus libero.

Generated 3 paragraphs, 343 words, 2347 bytes of Lorem Ipsum


\vfill
\clearpage

% ABSTRACT
% =========================================
%  \vspace{15cm}   make sapce on the top

{\normalfont
\color{uniblau}
\huge\sffamily\itshape
Abstract
}


In the traditional sense, the term typography refers to the design of printed works with movable letters (types). Initially, this was done in lead typesetting or wood typesetting.

In media theory, typography stands for printed type in contrast to handwriting (chirography) and electronic as well as non-literal texts.

Today, typography usually refers to the media-independent design process that uses type, images, lines, surfaces and empty spaces to create all kinds of communication media. In contrast to calligraphy, writing or type design, typography is the design with found material.



\vfill
\clearpage

% TABLE OF CONTENTS
% =========================================
\tableofcontents 
\clearpage

\pagenumbering{arabic} % Ab hier folgt die "arabische" Seitennummerierung. ( From here the "Arabic" pagenumbering follows.)

% Im Ordner chapter sollte für jedes Kapitel eine Datei angelegt und hier eingebunden werden.
% Das erhöht die Übersicht.

% ENG
% In the chapter folder a file should be created for each chapter and integrated here.
% That increases the overview.

% 1 INTRODUCTION
% ======================================
\chapter{1. Introduction}
% When there is more than one section
%\printmyminitoc{1} 

\section{Overview}
\section{Problem statement}
\section{Objectives of the research}
\section{How to use this template}
This is a free template for bachelor or master theses.\section {directory structure}

The actual contents of the thesis is placed at the directory: \texttt{chapter}. A separate file shall be created in the directory for each chapter. The separate chapters can then be integrated into a one file main.tex

Images should be uploaded to the \texttt {images} folder.


\subsection{Reference management}
References can be defined in Latex with \texttt {.bib} files and integrated into the running text with the command \texttt {\textbackslash {} cite} \cite {iuk696}. 

A reference management program makes the whole thing much easier, however, as it allows sources to be automatically imported, conveniently managed and cataloged. 

\subsection{Printing}

\begin{itemize}
    \item PRINT your paper for a cheaper price at \href{https://www.itmz.uni-rostock.de/anwendungen/multimedia/druckservice/}{Printing Service}
    \item Do the BINDING at a COPYshop (known as COPY and Paste) near Margaretenstraße 40 in Rostock.
\end{itemize}
\clearpage

% 2 THE ROLE OF SOFTWARE IN A RESEARCH
% ======================================
\chapter{The role of Software in Scientific research}
\printmyminitoc{1} % When there is more than one section


\section{Introduction }
Modern research is unthinkable without a use of software and scientific investigations in various areas of science are becoming increasingly reliant on software tools [1, 3 , 4, 14]. 
A software is very important asset for building a scientific knowledge and more discoveries in a research are made possible than ever by a use of software tools that automate processing of huge amount of data [9]. Typically a software is used in a research for control processes, simulation, modelling, data analysis, knowledge dissemination, etc. [2][25]. \\\\
Since software is not often considered as an academic output [24][25], it is usually not cited in research papers across several fields of research [25]. To counteract against this culture, a task force that advocates about the role of software in a research, known as Research software Alliance (ReSA), has been established. The ReSa promotes the inclusion of software as a primary research out put, influences decision makers to value a research software and give credits to the developers [21]. In 2019, the task force has collected literatures, at Zetoro group libraray, that evident significant roles of software in a research [22]. \\\\
Scientific software is often complicated and requires specialized domain knowledge for its development [3]. Due to this increasing number of scientists are developing a software as part of their research work or directly taking part in the development process of a research software [9, 15]. This fact is clearly reinforced by a  survey result from 2008 [1, 2, 6] which indicates that:

\vspace{-5mm}   %make sapce on the top
\begin{itemize}[noitemsep,topsep=0pt]
    \item nearly half of scientists spend more time developing a software as part of their research work than five years ago.
    \item 38 percent of researchers spend at least 20 percent of their time developing a software.
    \item over 90 percent of scientists agree that software is important for their research and
    \item nearly 3 out of 4 claim that their research directly depends on a use of a software.  
\end{itemize}

\section{General roles of software in a research }
Software is playing various crucial roles in a research and making a shift in a research culture. For example, software tools are making most of research to be increasingly data driven i.e. insights from an in-depth analysis of large volume of data-sets form the basis of a research conclusion [1][16].\\
Some of the most general roles of a software in a research are:
\vspace{-5mm}   %make sapce on the top
\begin{itemize}[noitemsep,topsep=0pt]
    \item Software helps to explore und understand a research problem [2].
    \item Results from a scientific software is presented as an evidence to support a research result [15]. 
    \item A use of (quality) software helps to produces better scientific discoveries [1]. A software dictates the quality of a research outcome[2] [23]. Outcome of a research becomes unreliable or even useless if there is an error in the software [5]. For example, several scientists retracted their scientific publications up on a retrospective discovery of a bug in their software [3, 6, 8]. A more palpable failure of a research ambition due to an error in the control-system software, for instance, is the failure of Ariane rocket in 1996 [17].  
    \item A software also helps to document a research process and to validate results of a given research [16]. Executable cells in a Jupyter notebook is one real world example where a software can be used to validate a research result.
\end{itemize}

\subsection{Role of software in specific domains of research }
A software is being extensively used for a research in various areas of science such as physics, chemistry, space science, life science and so on.\\\
The physics research facility, the Large Hydron Collider at CERN, for instance uses a software with more than 5 million lines of code which is used for processing of terabytes of data generated from experiments [4].\\\
In a nuclear research, a software is being developed increasingly to be used for experiments [19]. For example, testing a modification in a nuclear weapon can not be field tested, but instead a software that simulate the impact of modification is usually used [15]. This is because of regulations like nuclear test ban treaties and the potential disaster, to the environment and life, associated with nuclear weapons [20]. \\\
In chemistry research, a software can be used to model and simulate chemical processes that are challenging, too complex or expensive to conduct in reality. Karplus and Levitt used computer simulations for their joint-research “the development of multi-scale models for complex chemical systems”  and won a Nobel prize in 2013 for their work [4, 18].\\\
In a climate and environmental studies, software is used to make predictions about climate changes. For example a historical temperature data can be integrated to make predictions about future temperature variations [4].\\\
In a space science, space probes heavily rely on software. In this case a software navigates space crafts to other planets, processes and transmits scientific data back to Earth fur more processing, helps researchers interpret results, etc[28].
\subsection{The role of software in research breakthroughs}
Several research breakthroughs has been made possible because of the use of software in the research. \\
One of the research breakthroughs is creation of the very first visual representation of a black hole using an open source software NumFOCUS. To observe a black hole that is 55 million light years away, it would have required to build a huge telescope of size of planet earth. But instead of building one giant telescope, hundreds of scientists spent decades of years creating a global network of telescopes, known as Event Horizon Telescope (EHT) [29], synchronized precisely using atomic clocks. The EHT gathered a huge amount of data for years. However there was a lot of noise in data the collected data because :

\vspace{-5mm}   %make sapce on the top
\begin{itemize}[noitemsep,topsep=0pt]
    \item The EHT was a network of non-similar telescopes.
    \item The radio signals were coming through attenuated due to atmospheric effect like water vapor, clouds, turbulence … etc.
\end{itemize}

Therefore the scientists had to use various algorithms and data analysis pipelines. The resulting image from various data processing was compared to ensure the integrity of the result. This huge scientific breakthrough in a space research, can be attributed to mainly the use of powerful data processing software. 
Other scientific breakthroughs that can be attributed to role of software in a research include:
\vspace{-5mm}   %make sapce on the top
\begin{itemize}[noitemsep,topsep=0pt]
    \item The detection and visualization of gravitational waves for the first time, using a LIGO software [30][31]. 
    \item Software accelerates drug discovery [32].
\end{itemize}

\section{Summary}
Overall the role of software in a research are:
\vspace{-5mm}   %make sapce on the top
\begin{itemize}[noitemsep,topsep=0pt]
    \item software assists in the exploration a research problem. 
    \item results from a scientific software form an evidence for a research conclusion.
    \item software assists for better scientific discoveries and research breakthroughs. 
    \item research software helps to document the steps followed during a research, which consequently, help to validate or reproduce the research result.
    
\end{itemize}
\clearpage


% 3 SOFTWARE USAGE PURPOSE
% ======================================
\input{chapter/6. purpose of software}
\clearpage

\listoffigures % Abbildungsverzeichnis ( List of figures)
\printbibliography % Quellenverzeichnis (Bibliography)
\clearpage

\end{document}