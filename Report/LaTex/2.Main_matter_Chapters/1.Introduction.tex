% --------------------------------------------------------------------
% SECTION - I - INTRODUCTION
% --------------------------------------------------------------------
\section{Introduction}

\subsection{Recap}

According to the researh conducted by \cite{}

\begin{itemize}
    \item Analysis of research papers can give a lot of insights about software resources and their dependency. 
    \item In a scientific research different kinds of input resources are used. One of such input is a software. 
    \item Used resources in a research are typically mentioned in a citation. Citation practices of formal articles in a research are matured and various citation styles exist. Even if principles  for formal citation of a software has already been put out, most scientists are not properly citing resources.  
	\item Surprisingly, sometimes researchers do not mention the type of software they used entirely or mention it with a abbreviation ... not enough creadit is being given to research software developers .
	\item As long as software is mentioned using formal methods, like RRID, it is possible to perform citation analysis using regular expressions which can be constructed to capture the pattern of citation.
	\item hough regular expression based analysis can give basic insights about the software citation it has limitations because:
			
% sub-bullets------------------------------------------------------------------------
			\begin{itemize}
				\item Not so many authors use formal citation of software, like RRIDs
				\item Even if scientists use formal citations, they may fail to properly follow the guidelines. For example, some authors tend to ignore the RRID-part and that creates an ambiguity by it self that it is not possible to know weather the author is actually making a software citation or it is completely something else. 
				\item Rule based method fails to capture context information and ignores dependencies. It is not possible to be sure about the authors intention whether or not using a software citation. 

			\end{itemize}
% end sub-bullets------------------------------------------------------------------------

	\item At the same time pattern based analysis, like using regX, is not suitable to extract information about software citation, for instance the particular use of a software, especially when a software mention statement lacks any form of formality where the information is concealed in a natural language description. 
	\item Therefore it is required to automatically extract the purpose of software use in scientific literatures. This might help to answer questions like:
		
% sub-bullets------------------------------------------------------------------------
		\begin{itemize}
			\item What type of software is being frequently used for what purpose in a specific area of research? This also allows to find an answer further question like what is  the most common technique researchers follow when trying to solve a given research problem in a given domain )


		\end{itemize}	
% end sub-bullets------------------------------------------------------------------------
	\item Previous attempts to automatically extract information using machine learning techniques, specifically supervised machine learning technique, about the software use purpose was constrained mainly because of lack of ground truth data. But this time, with the advent of SoMeSci, it is possible to do so. 	
	
\end{itemize}


\subsection{Problem Satement}
\subsection{Objectives of this project}
This project has the following objectives:
		
%\vspace{-5mm}   %make sapce on the top
\begin{itemize}[noitemsep,topsep=5pt]   % decrease the space between bullets

	\item List down the purpose of software usage .
	\item To extend SoMeSci with annotation of purpose of software usage. 
	\item To select feature for the training model
	\item To select a classifier and train a model.
	\item To evaluate and optimize results 

\end{itemize}	

\clearpage
