\chapter{Introduction}
\label{ch:intro}

Today’s scientific research relies on a use as well as production of various types of research artifacts, a.k.a. research resources, ranging from digital to physical \citep{schindler2019annotation}. One of key digital research artifacts, broadly used in a scientific investigations, is software. Many scientists use already existing software for various purposes during their research as well as create a new software as part of their research work \citep{goble2014better, hannay2009scientists}. \\

Almost every software has information associated with it which can be extracted. Version, name of developer, license, abbreviation, URL, citation, release, extension, etc. are among the most obvious examples.  \\

However there are also other kinds of information about a software that is non-obvious and challenging to extract. For  example, in a given line of text in a research paper a software can be mentioned to indicate weather a researcher is using a software for a particular purpose, introducing a novel software, providing repository information about the deposition of the software, or just mentioning the name of the software. In addition information about category of a software: such as application, plugin, programming environment, operating system, etc.  can also be concealed in a textual description in a scientific paper \citep{schindler2021somesci}. \\

Extraction of all such variations of information about software from the scientific publications has critical importance. This is because, information about a software can be used to uniquely identify each software and avoid ambiguity regarding which software or version a researchers have used in their literature. Being able to uniquely identify a software with its specification is also advantageous to guarantee reproducibility of research results as well as providing clear understanding how results of research have been produced \citep{kruger2019literature}. In addition, knowledge about a software mention type or purpose of use can help to determine which set of software artifacts can be suitable for a given study or to compare results obtained from various software in a given study. Furthermore, knowledge about software use and purpose of use in the literature, supports semantic analysis and retrieval of scientific publications based on use of particular software \citep{schindler2019annotation}. \\

Even though software citation principles have been already established by a scientific community \citep{callaghan2014joint, smith2016software}, software citation practice in reality  is still informal and incomplete \citep{schindler2021somesci}. This makes it difficult to extract information about software that would help to attribute credits to the creators of software, reproduce research results, disambiguate one software from another, etc. \\

Various manual and rule based techniques has been attempted in the past to extract information about software. However, machine learning based techniques have not been exploited despite being more powerful. The main reason was lack of training data which can support training of a classifier for software information extraction \citep{schindler2021somesci}. Producing reliable ground truth data could be accomplished by crowd sourcing for general domains but it is expensive particularly for domain-specific and scientific publications as it requires expert domain knowledge \citep{beltagy2019scibert}. Fortunately, identification of software mentions from scientific articles has drawn more attention over the past years and now various labeled data sets, such as BioNerDs \citep{duck2013bionerds} , SoftCite \citep{du2021softcite}, are available. Recently a more comprehensive data set, SoMeSci, has been published.  SoMeSci contains high quality manually annotated data sets that cover broader range of information about software paving a way for a use of machine learning based approach for the automatic extraction of information about software \citep{schindler2021somesci}. 



\section{Scope}
\label{sec:intro:Scope}

This thesis work tries to apply machine learning technique using SoMeSci data set to extract information about software mentions, particularly, to identify for what purpose a given software is used in a given literature. \\

To accomplish this, first possible list of software usage purposes have been identified via extensive analysis of literature and other sources like software ontologies and repositories. Then already existing annotations of software usage mentions in the SoMeSci data set has been extended with software purpose labels. \\

Once software usage mentions in the SoMeSci data set has been labelled with software purpose labels, the data set has been cleaned, analyzed, transformed, and used for classification purpose.


\section{Objectives of the research}
\label{sec:intro:Objectives}
The main objectives to be accomplished in this thesis work are:


\begin{itemize}
    \item To perform literature review on the importance of software in a research.
    \item To carry out analysis of literature and software ontologies to identify main types of purposes of software use in a research.
    \item	To extend SoMeSci data set with software usage purpose annotation. 
    \item	To perform analysis of SoMeSci data set to drive interesting facts about the data set. 
    \item	To select a suitable feature, classifier, and to train a classifier model.
    \item	To evaluate and optimize results.
	
\end{itemize}


%
% Section:  Overview of the report 
%
\section{ Overview of the report }
\label{sec:intro:Overview}

  Chapter 1 makes a soft introduction about why it is important to extract information about software, specifies scope and objective of the thesis. \\

Chapter 2 focuses on highlighting the role of software in a research to indicate driving information about software used in a research is an important task. \\

Chapter 3 focuses to identify possible types of software usage purposes from literature and software ontology. This is an important step taken to extend software usage statements in the SoMeSci data set with software purpose annotations. \\

Chapter 4 is about the data set.  It explains how SoMeSci data set has been extended with software purpose annotations, annotation tool used, and the annotation process. In addition explains about data pre-processing, transformation to suitable format and splitting for classification on purpose. At the end, results of analysis of the extended SoMeSci data set has been presented. \\

Chapter 5 discusses and compares a various models suitable for  classification of software purpose from a text. \\

Chapter 6 focuses on selection of feature for the classifier. \\

Chapter 7 discusses about classification and results  of the evaluation. \\

Chapter 8 summarizes results and provides conclusion.


\section{Summary}
\label{sec:intro:Summary}

This chapter has presented a gentle introduction into types of information associated with software artifacts, extraction approaches and why it is important to extract information about software. The data set to be used, scope and goal of the work has also been discussed. 
The next chapter presents the role of software in a modern research to give more elaborate understanding about the impact of software in a scientific investigations indicating why it is worth to extract knowledge about software.  

\clearpage
