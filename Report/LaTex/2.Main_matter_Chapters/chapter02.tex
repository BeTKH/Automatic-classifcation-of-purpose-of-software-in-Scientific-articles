\chapter{The role of Software in Scientific research}
\label{ch:Roles}

%
% Section: Intro

%-----------------------------
\section{Introduction}
%-----------------------------
\label{sec:background:intro}
Software is a collection of instructions that supervise a computer how to execute a given task \citep{enwiki:1056292826}. The behavior of such instructions is specified by algorithms which are derived from scientific laws \citep{wolfram1984computer}.  Implementation of algorithms is carried out using programming languages, the end result being a software \citep{enwiki:1055624679, enwiki:1055665216}. 

Modern research is unthinkable without a use of software and scientific investigations in various areas of science are becoming increasingly reliant on software tools \citep{goble2014better, wilson2014best, storer2017bridging}.

A software is very important asset for building a scientific knowledge and more discoveries in a research are made possible than ever by a use of software tools that automate processing of huge amount of data \citep{jimenez2017four}. Typically a software is used in a research for control processes, simulation, modelling, data analysis, knowledge dissemination, etc. \cite{hannay2009scientists, pan2016disciplinary}.

Since software is not often considered as an academic output \citep{yang2018important, pan2016disciplinary}, it is usually not cited in research papers across several fields of \cite{pan2016disciplinary}. To counteract against this culture, a task force that advocates about the role of software in a research, known as \ac{ReSA}, has been established. The ReSa promotes the inclusion of software as a primary research out put, influences decision makers to value a research software and give credits to the developers \footnote{https://www.researchsoft.org/about-resa/}. In 2019, the task force has collected literatures, at \href{https://www.zotero.org/groups/2400609/resa/library}{Zetoro group libraray} , that evident significant roles of software in a research \footnote{https://doi.org/10.5281/zenodo.3884311}.  



Scientific software is often complicated and requires specialized domain knowledge for its development \citep{wilson2014best}. Due to this, increasing number of scientists are developing a software as part of their research work or directly taking part in the development process of a research software \citep{jimenez2017four, kanewala2014testing}. This fact is clearly reinforced by a  survey results  from 2008, 2014, and 2017  in the \ac{UK} and \ac{USA}. Participants of the survey were ….. around 2000 academic staff, postdocs, Tas… \citep{merali2010computational, hettrick2014uk, nangia2017track}.  The results indicate that:


\begin{itemize}%[noitemsep,topsep=5pt, leftmargin=0.3in] % decrease the space between bullets
	\item Almost half of scientists spend more time developing a software as part of their research work than five years ago.
	\item 38\% of researchers spend at least 20\% of their time developing a software.
	\item Over 90\% of scientists say software is important for their research \&
	\item Nearly 70\% claim that their research directly depends on a use of a software.  

\end{itemize}




%-----------------------------
\section{General roles}
%-----------------------------
\label{subsec:background:first_section:first_subsection}

Software is playing various crucial roles in a research and making a shift in a research culture. For example, software tools are making most of research to be increasingly data driven i.e. insights from an in-depth analysis of large volume of data-sets form the basis of a research conclusion \citep{goble2014better, jay2020software}. Some of the most general roles of a software in a research are:

\begin{itemize}%[noitemsep,topsep=5pt, leftmargin=0.5in] % decrease the space between bullets

	\item Software helps to explore und understand a research problem \citep{hannay2009scientists}.
	\item Results from a scientific software is presented as an evidence to support a research result \citep{kanewala2014testing}. 
	\item A software dictates the quality of a research outcome \citep{hannay2009scientists}. Outcome of a research becomes unreliable or even useless if there is an error in the software \citep{soergel2014rampant}. For example, several scientists retracted their scientific publications up on a retrospective discovery of a bug in their software \citep{wilson2014best, merali2010computational, miller2006scientist}. A more palpable failure of a research ambition due to an error in the control-system software, for instance, is the failure of Ariane rocket in 1996 \citep{enwiki:1054482061}.  
	\item A software also helps to document a research process and to validate results of a given research \citep{jay2020software}. Executable cells in a Jupyter notebook is one real world example where a software can be used to validate a research result.
	\item Software allows experiments to be made beyond constrains of the physical world. This is because experiments that run on a computer are not limited by processes that occur in nature but only by the laws imbedded in the computer code \citep{wolfram1984computer}. 

\end{itemize}

%-----------------------------
\section{Domain Specific Examples}
%-----------------------------
\label{subsec:background:first_section:second_subsection}
A software is being extensively used for a research in various areas of science such as physics, chemistry, space science, life science and so on.

The physics research facility, the \ac{LHC}\ at \ac{CERN}, for instance uses a software with more than 5 million lines of code which is used for processing of terabytes of data generated from experiments \citep{storer2017bridging}.
In a nuclear research, a software is being developed increasingly to be used for experiments \citep{yan2017case}. For example, testing a modification in a nuclear weapon can not be field tested, but instead a software that simulate the impact of modification is usually used \citep{kanewala2014testing}. This is because of regulations like nuclear test ban treaties and the potential disaster, to the environment and life, associated with nuclear weapons \citep{enwiki:1053274189}. 

In chemistry research, a software can be used to model and simulate chemical processes that are challenging, too complex or expensive to conduct in reality. Karplus and Levitt used computer simulations for their joint-research “the development of multi-scale models for complex chemical systems”  and won a Nobel prize in 2013 for their work \citep{storer2017bridging, andre2014nobel}. 

In a climate and environmental studies, software is used to make predictions about climate changes. For example a historical temperature data can be integrated to make predictions about future temperature variations \citep{storer2017bridging}.

In a space science, space probes heavily rely on software. In this case a software navigates space crafts to other planets, processes and transmits scientific data back to Earth fur more processing, helps researchers interpret results, \citep{lutz2011software}. 

Software, specifically imaging software, plays a critical role to assist medical researchers for early isolation of cancer and ultimately to saving life.  The main reason for low chance of survival from cancer is mainly due to late detection of cancer cells in the body  and once cancer spreads throughout the body it is difficult to treat. This makes a diagnosis of cancer to be a time critical task and early identification of cancer implies curability of a disease and a higher chance of survival \citep{wagner2004challenges}. Especially on the early stages, it is not straight forward to determine which cells are likely to develop a cancer. For this reason, medical scientists use different types of software to identify cancer cell or to decide weather a tumor is malignant or not. Using a software, they could perform various kinds of analysis and processing on imageries obtained from scans such as \ac{MRI} or \ac{CT} Scan \citep{al2012lung}. An example of software that is used for cancer imaging research is DMRI. Such software is extensively used by many researchers, more than 75,000 downloads every year \citep{norton2017slicerdmri}. Therefore, it is reasonable to say that software helps to save life.

Software plays an important role in power system planning and operation. One of the major activities in power system operation is contingency analysis. During contingency analysis, engineers determine violations of power grid operation conditions, such as overloading, which might occur when outage of a transmission line or a power generation unit occurs. Contingency analysis helps to understand power system behavior after outages and gives an opportunity to take preventative actions \citep{mishra2012contingency}. Power grids are extremely complex and such kind of analysis tasks are unimaginable with out a use of software. An example of software that is used to perform contingency analysis in the power system operation is \emph{Power World} software.  

%
% Section:The Role in Research Breakthroughs

%-----------------------------
\section{The Role in Research Breakthroughs  }
%-----------------------------
\label{sec:background:second_section}
A use of software also allowed to produce better scientific discoveries and several research breakthroughs has been made possible\citep{goble2014better}. 

One of the research breakthroughs is creation of the very first visual representation of a black hole using an open source software NumFOCUS \footnote{numfocus.org}. To observe a black hole that is 55 million light years away, it would have required to build a huge telescope of size of planet earth. But instead of building one giant telescope, hundreds of scientists spent decades of years creating a global network of telescopes, known as \ac{EHT} \citep{enwiki:1052167868}, synchronized precisely using atomic clocks. The \ac{EHT} gathered a huge amount of data for years. However there was a lot of noise in the collected data because :


\vspace{-1mm}   %make sapce on the top
	\begin{itemize}%[noitemsep,topsep=5pt, leftmargin=0.5in] % decrease the space between bullets
		\item The \ac{EHT} was a network of non-similar telescopes.
		\item The radio signals were coming through attenuated due to atmospheric effect like water vapor, clouds, turbulence … etc.
	\end{itemize}
Therefore the scientists had to use various algorithms and data analysis pipelines. The resulting image from various data processing was compared to ensure the integrity of the result. This huge scientific breakthrough in a space research, can be attributed to mainly the use of powerful data processing software. 

Other scientific breakthroughs that can be attributed to software use in a research include:

	\vspace{-1mm}   %make sapce on the top
	\begin{itemize}%[noitemsep,topsep=5pt, leftmargin=0.5in] % decrease the space between bullets
		\item The detection and visualization of gravitational waves for the first time, using a \ac{LIGO} software \citep{enwiki:1047100294}\footnote{https://pegasus.isi.edu/2017/10/19/pegasus-contributed-to-new-gravitational-wave-detector-discovery}. 
		\item Software accelerates drug discovery \footnote{https://phys.org/news/2018-08-software-framework-drug-discovery-ieee.html}.
	\end{itemize}


