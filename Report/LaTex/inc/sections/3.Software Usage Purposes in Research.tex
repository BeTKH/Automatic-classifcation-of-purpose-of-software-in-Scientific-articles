\section{Literature Review on Software Usage Purpose in Research }
\subsubsection{Introduction}
Scientists use various kinds of software, during their research, for different purposes. Some times software is used for execution of some trivial tasks like word processing and in other cases they use a software to perform critical tasks that can ultimately determine their research end result. 

\subsubsection{Software usage purposes in a research}
In a modern research, where a research is increasingly relying on processing of huge amount of data, the most common purpose of software usage purpose is to perform data analysis. 

Data analysis is a broad term which can refer to inspecting, cleaning, transforming, modelling data, etc. with a particular goal of discovering a meaningful information from the data which can be used to make conclusions or decisions [13] . 

When it comes to the application of data analysis in actual research works, various kinds of data analysis techniques exit. Some of the data analysis techniques can be more general where as others are more domain specific. Some common examples of software usage purposes:
\vspace{-1mm}   %make sapce on the top
	\begin{itemize}[noitemsep,topsep=5pt, leftmargin=0.5in] % decrease the space between bullets
		\item Data Analysis, Mathematical Analysis, Statistical Analysis, Numerical Analysis, Text Analysis
		\item Domain specific Analysis e.g. Densitometric Analysis, Voxel-based Analysis
		\item Data Processing , e.g. Image processing
		\item Data Collection
		\item Modelling
		\item Simulation
		\item Programming
		
	\end{itemize}

\clearpage