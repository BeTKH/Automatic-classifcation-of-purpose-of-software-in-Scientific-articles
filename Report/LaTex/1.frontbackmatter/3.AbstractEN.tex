%*******************************************************
% Abstract in English
%*******************************************************
\pdfbookmark[0]{Abstract}{Abstract}


\begin{otherlanguage}{american}
	\chapter*{Abstract}

	\medskip
	
	\noindent

\end{otherlanguage}

These days scientific research is increasingly dependent on software as many scientists not only use existing software, but also create one to use in their research process. This strong reliance on software, besides its benefits, has also created issues regarding reliability and quality of research results from a software. Some scientists were forced to retract their scientific work upon a retrospective discovery of an error in a software they used, which rendered their research worthless. In addition, use of software for scientific research entailed difficulty of reproducing research results mainly because of poor software citation practices. \\

This thesis contributes to an effort to alleviate the above problems by extracting useful information about software tool particularly the purpose of use of software in a scientific paper. Automatic extraction of purpose of software usage creates an opportunity to identify a set of software tools used for the same purpose in a given context of scientific work. This in turn helps to compare results obtained among similar software tools enabling evaluation for quality and reliability of research software. In addition, automatic extraction of software usage purposes enables to semantically browse related research articles based on their use of a specific software for the same purpose. \\

In this work, SoMeSci data set has been annotated with labels that indicate the purpose of software use which have been identified by extensive research of literature and software ontologies. Further, Sci-BERT classifier model has been  selected and trained on SoMeSci dataset after exploring various kinds of models that suit for extraction of software usage purposes. The classifier model has been further optimized by considering various training scenarios such as more context information, inclusion or exclusion of some parts of SoMeSci data set, simplifying classifier model with fewer classifier modules. Overall, results of model training, indicates the use of classifier models pre-trained on scientific corpus, such as Sci-BERT and Bio-BERT, enabled automatic classification of software usage purposes with a reasonable performance despite a limited amount of labeled dataset. Further more, it was observed that consideration of broader context, such as 2 adjacent sentences, improved classifier model’s performance.
